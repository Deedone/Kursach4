\documentclass[a4paper, ukrainian, utf8, simple]{eskdtext}
\usepackage[T1,T2A]{fontenc}
\usepackage{geometry}
\usepackage[utf8]{inputenc}



\ESKDdepartment{Акционерное Московское общество}
\ESKDcompany{завод имени И.~А.~Лихачева}
\ESKDtitle{Маршрутизація в мережі передачі даних}
\ESKDdocName{Пояснювальна записка}
\ESKDsignature{ІАЛЦ.467100.002 ПЗ}
\ESKDauthor{Потурай М.В.}
\ESKDcolumnIX{\small КПІ ім. Ігоря Сікорського %
ФПМ %
КВ-71}
\begin{document}

\maketitle

\newgeometry{left=2cm, right=2cm}
\newpage
\ESKDthisStyle{empty}
\begin{center}
\textbf {Анотація}
\end{center} 

В даному курсовому проекті розроблена програма для моделювання комп'ютерної мережі.
Програма дозволяє аналізувати різні режими та час передачі повідомлень, кількість управляючих та повторних пакетів та інше.
Розроблений алгоритм дозволяє моделювати мережу, наближену до реальної.
Також розроблений зручний графічний інтерфейс на мові програмування Typescript.

\newpage
\ESKDthisStyle{empty}
\begin{center}
\textbf {Annotation}
\end{center} 

The subject of this course work is modelling of the computer networks.
Thus project includes a program which allows to analyze different transmitting modes, the time of message transmitting, number of control packets and repeatedly sent packets etc.
The algorithm of presented program allows to make a close to reality simulation;
It is supplemented with a user-friendly graphical interface made by means of a Typescript programming language;
\restoregeometry
\newpage
\ESKDthisStyle{formII}



\tableofcontents
\newpage

\section{Вступ}
Появу комп'ютерних мереж можна розглядати як важливий крок у розвитку
комп'ютерної техніки. Вони дозволили об'єднувати багато комп'ютерів в
одну велику структуру, та значно збільшили швидкість передачі інформації
на великі відстані. Це зробило можливим співпрацю людей із різних частин 
світу, що значно прискорило науково-технічній прогрес.

Наразі більшість мешкінців цивілізованного світу не можуть представити 
своє життя без глобальної мережі Internet. До неї підключені мільйони 
пристроїв по всьому світу. Саме через таку величезну кількість пристроїв 
необхідно щоб алгоритми маршрутизації, що керують мережею, були максимально
ефективними. Але не існує одного ідеального алгоритму. Для мереж із різною 
топологією оптимальними будуть різні алгоритми, тож перед побудовою мережі
дуже корисно мати засоби для моделювання та перевірки алгоритмів.

Даний проект дозволяю змоделювати поведінку мережі довільної структури.
Це дозволить швидко перевірити придатність обраного алгоритму для керування
заданим типом мережі, щоб уникнути помилок та зайвих витрат. Проект має 
досить гнучкі налаштування там зручний інтерфейс користувача. Для зменшення
складності проекту, різноманіття фізичних параметрів апроксимується приблизним
значенням. 

\end{document}
