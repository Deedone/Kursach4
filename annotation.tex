\newgeometry{left=2cm, right=2cm}
\newpage
\ESKDthisStyle{empty}
\begin{center}
\textbf {Анотація}
\end{center} 

В даному курсовому проекті розроблена програма для моделювання комп'ютерної мережі.
Програма дозволяє аналізувати різні режими та час передачі повідомлень, кількість управляючих та повторних пакетів та інше.
Розроблений алгоритм дозволяє моделювати мережу, наближену до реальної.
Також розроблений зручний графічний інтерфейс на мові програмування Typescript.

\newpage
\ESKDthisStyle{empty}
\begin{center}
\textbf {Annotation}
\end{center} 

The subject of this course work is modelling of the computer networks.
Thus project includes a program which allows to analyze different transmitting modes, the time of message transmitting, number of control packets and repeatedly sent packets etc.
The algorithm of presented program allows to make a close to reality simulation;
It is supplemented with a user-friendly graphical interface made by means of a Typescript programming language;
\restoregeometry
\newpage
\ESKDthisStyle{formII}

