\newpage
\newgeometry{left=2cm, right=2cm}
\ESKDthisStyle{empty}
\section*{Технічне завдання}

\renewcommand{\theenumi}{\arabic{enumi}}
\begin{enumerate}
    \item Розробити програму, яка дозволяла б моделювати процес визначення  маршруту передачі повідомлень в мережі передачі даних заданої топології (конфігурації) та передачу повідомлень в режимі встановлення логічного з’єднання та дейтаграмному режимі.
  \item Знайти найкоротші шляхи та маршрути з мінімальним числом транзитних ділянок.
  \item Представити таблиці відстаней та маршрутів у кожному вузлі мережі передачі даних.
  \item Визначити час доставки повідомлень.
  \item Визначити кількість управляючих повідомлень і загальну кількість повідомлень, необхідних для передачі даних при встановленні логічного з’єднання між вузлами мережі та при передачі в дейтаграмному режимі.
  \item Порівняти результати і зробити висновки.
\end{enumerate}
\subsection*{Топологія мережі за варіантом}
3 регіональні мережі, кожна з яких складається мінімум з 11 комунікаційних вузлів.
1 канал - супутниковий.
Середня ступінь мережі - 4.
Кожен третій вузол комуникаційний.
Ваги каналів обираються випадковим чином із значень 3, 6, 7, 8, 12, 15, 18, 21, 22, 24, 30, 32.
Використовується алгоритм дельта-маршрутизації.
\newgeometry{left=3cm, right=3cm}
